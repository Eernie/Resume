%%%%%%%%%%%%%%%%%%%%%%%%%%%%%%%%%%%%%%%
% Deedy - One Page Two Column Resume
% LaTeX Template
% Version 1.1 (30/4/2014)
%
% Original author:
% Debarghya Das (http://debarghyadas.com)
%
% Original repository:
% https://github.com/deedydas/Deedy-Resume
%
% IMPORTANT: THIS TEMPLATE NEEDS TO BE COMPILED WITH XeLaTeX
%
% This template uses several fonts not included with Windows/Linux by
% default. If you get compilation errors saying a font is missing, find the line
% on which the font is used and either change it to a font included with your
% operating system or comment the line out to use the default font.
% 
%%%%%%%%%%%%%%%%%%%%%%%%%%%%%%%%%%%%%%
% 
% TODO:
% 1. Integrate biber/bibtex for article citation under publications.
% 2. Figure out a smoother way for the document to flow onto the next page.
% 3. Add styling information for a "Projects/Hacks" section.
% 4. Add location/address information
% 5. Merge OpenFont and MacFonts as a single sty with options.
% 
%%%%%%%%%%%%%%%%%%%%%%%%%%%%%%%%%%%%%%
%
% CHANGELOG:
% v1.1:
% 1. Fixed several compilation bugs with \renewcommand
% 2. Got Open-source fonts (Windows/Linux support)
% 3. Added Last Updated
% 4. Move Title styling into .sty
% 5. Commented .sty file.
%
%%%%%%%%%%%%%%%%%%%%%%%%%%%%%%%%%%%%%%%
%
% Known Issues:
% 1. Overflows onto second page if any column's contents are more than the
% vertical limit
% 2. Hacky space on the first bullet point on the second column.
%
%%%%%%%%%%%%%%%%%%%%%%%%%%%%%%%%%%%%%%

\documentclass[]{deedy-resume-openfont}


\begin{document}

%%%%%%%%%%%%%%%%%%%%%%%%%%%%%%%%%%%%%%
%
%     TITLE NAME
%
%%%%%%%%%%%%%%%%%%%%%%%%%%%%%%%%%%%%%%


\namesection{Erwin}{Oldenkamp}{ \urlstyle{same}\url{http://www.eernie.nl} | 21-02-1991 \\
\href{mailto:erwin.oldenkamp@eernie.nl}{erwin.oldenkamp@eernie.nl} | 0627481354
}

%%%%%%%%%%%%%%%%%%%%%%%%%%%%%%%%%%%%%%
%
%     COLUMN ONE
%
%%%%%%%%%%%%%%%%%%%%%%%%%%%%%%%%%%%%%%

\begin{minipage}[t]{0.25\textwidth} 	
	
\section*{}
\vspace{-0.3in}
\includegraphics[width=0.8\textwidth]{foto-croped}
		
%%%%%%%%%%%%%%%%%%%%%%%%%%%%%%%%%%%%%%
%     Address
%%%%%%%%%%%%%%%%%%%%%%%%%%%%%%%%%%%%%%

\section{Adres}
\descript{Telemannstraat 277}
\descript{8031 KJ Zwolle}
\sectionsep
	
%%%%%%%%%%%%%%%%%%%%%%%%%%%%%%%%%%%%%%
%     LINKS
%%%%%%%%%%%%%%%%%%%%%%%%%%%%%%%%%%%%%%

\section{Links} 
Github:// \href{https://github.com/eernie}{\custombold{eernie}} \\
StackOverflow:// \href{http://stackoverflow.com/users/1956445/eernie}{\custombold{eernie}}\\
LinkedIn://  \href{https://www.linkedin.com/in/erwinoldenkamp}{\custombold{erwinoldenkamp}}\\
Dockerhub:// \href{https://hub.docker.com/u/eernie}{\custombold{eernie}}\\
Maven:// \href{https://search.maven.org/#search\%7Cga\%7C1\%7Cnl.eernie}{\custombold{nl.eernie}}
\sectionsep

%%%%%%%%%%%%%%%%%%%%%%%%%%%%%%%%%%%%%%
%     Certficates
%%%%%%%%%%%%%%%%%%%%%%%%%%%%%%%%%%%%%%

\section{Certificaten}
\descript{2015 | Java SE 7 OCA }
\descript{2016 | Java SE 7 OCP }
\descript{2017 | Java SE 8 OCP }
\sectionsep

%%%%%%%%%%%%%%%%%%%%%%%%%%%%%%%%%%%%%%
%     HOBBIES
%%%%%%%%%%%%%%%%%%%%%%%%%%%%%%%%%%%%%%
\section{Hobbies}
\custombold{Fitness}\\
\custombold{Drummen}\\
\custombold{Muziek}\\
\custombold{Opensource projects}\\
\custombold{Reizen}\\
\custombold{Fotografie}\\
\sectionsep


%%%%%%%%%%%%%%%%%%%%%%%%%%%%%%%%%%%%%%
%
%     COLUMN TWO
%
%%%%%%%%%%%%%%%%%%%%%%%%%%%%%%%%%%%%%%

\end{minipage} 
\hfill
\begin{minipage}[t]{0.74\textwidth} 

%%%%%%%%%%%%%%%%%%%%%%%%%%%%%%%%%%%%%%
%     EXPERIENCE
%%%%%%%%%%%%%%%%%%%%%%%%%%%%%%%%%%%%%%
\section{Werkervaring}

\runsubsection{Getthere - TenneT TSO}
\descript{| Java Engineer}
\location{Januari 2018 - Heden | Arnhem}
\vspace{-2mm}
\begin{itemize} \setlength\itemsep{0.1em}
\item In een team tiental applicaties in beheer, waarvan de meeste lagacy applicaties zijn en worden uitgefaseerd naar een nieuw landschap
\item Nieuwe projecten gebouwd in een Java EE container in combinatie met Angular
\item Migraties van SVN naar GIT verzorgd voor de legacy applicaties 
\item Gerealiseerd dat alle applicaties in beheer door het team met een uniform Jenkinsfile gebouwd worden
\item Verantwoordelijkheden: Ontwikkelen van de applicaties, innovatie doorvoeren in de huidige applicatie en kennis overdracht van Angular
\end{itemize}
\vspace{-4mm}
\textsubscript{Agile \textbullet{} Java EE \textbullet{} Jboss EAP \textbullet{} Weblogic \textbullet{} Oracle \textbullet{} Docker \textbullet{} Fitnesse \textbullet{} Jenkins 2.0 \textbullet{} NgRx \textbullet{} Html/Css \textbullet{} Full-stack}
\sectionsep

\runsubsection{Topicus Legal}
\descript{| Software Engineer / Architect }
\location{April 2016 - December 2017 | Zwolle}
\vspace{-2mm}
\begin{itemize} \setlength\itemsep{0.1em}
\item In een klein team gewerkt aan de applicatie van de nieuwe vertical van Topicus. 
\item Groot gedeelte van het landschap ontworpen en vervolgens met 2 andere ontwikkelaars de complete applicatie gebouwd.
\item Het platform is active-active opgezet en draait gecontaineriseerd door middel van docker. 
\item De backend is volledig geschreven in Java en draait op het Java EE platform.
\item De frontend is ontwikkeld in Angular 2/4+. 
\item Verantwoordelijkheden: Architectuur, ontwikkelen fullstack, opzetten van schaalbare bouwstraat en service afdeling die de applicaties in productie beheert ondersteunen.
\end{itemize}
\vspace{-4mm}
\textsubscript{Agile \textbullet{} Java EE\textbullet{} Wildfly \textbullet{} Angular \textbullet{} Docker \textbullet{} Cucumber \textbullet{} PostgreSQL \textbullet{} Jenkins 2.0 \textbullet{} Html/Css \textbullet{} Full-stack}
\sectionsep

\runsubsection{Topicus Finan B.V.}
\descript{| Software Engineer }
\location{Februari 2014 - 2016 | Zwolle}
\vspace{-2mm}
\begin{itemize} \setlength\itemsep{0.1em}
\item In een team van 10 man mee geholpen aan een backoffice applicatie die op hoog volume orders kan verwerken. 
\item De applicatie is veelzijdig inzetbaar. 
	Het kan gebruikt worden als enkel een facturatiesysteem, zoals gedaan voor enkele ziekenhuizen, als een backoffice voor hypotheken beheer en incassatie, of zelfs 
	als complete backoffice voor één van de grootste retailers in Nederland. 
\item De applicatie draait op meerdere Java EE servers namelijk Websphere en Wildfly, met een database Oracle of Postgres.
\item Communicatie over SOAP+JMS of SOAP+HTTPS 
\item De grootste uitdaging was het ondersteunen van de bovengenoemde opties.
\item Verantwoordelijkheiden: Ontwikkelen applicaties en innovatie doorvoeren waar mogelijk
\end{itemize}
\vspace{-4mm}

\textsubscript{Agile \textbullet{} Java EE\textbullet{} Spring \textbullet{} Wicket \textbullet{} wildfly \textbullet{} Websphere \textbullet{} Oracle \textbullet{} PostgreSQL \textbullet{} JMS}
\sectionsep

\runsubsection{Inversive Media}
\descript{| Software Developer }
\location{Juli 2013 - Augustus 2013 | Groningen}
\vspace{-2mm}
\begin{itemize} \setlength\itemsep{0.1em}
\item Gewerkt aan een Java applicatie geschreven in Play met een PosgreSQL database genaamd `Tafelreserveren.nl`
\item Zoekpagina geschreven in combinatie met ElasticSearch
\item Verantwoordelijkheden: Ontwikkelen applicatie, oplossen van bugs en bouwen van nieuwe features
\end{itemize}
\vspace{-4mm}
\textsubscript{Java\textbullet{} Play framework \textbullet{} ObjectiveC \textbullet{} PostgreSQL \textbullet{} ElasticSearch}
\sectionsep

\end{minipage} 
\pagebreak

\begin{minipage}[t]{0.59\textwidth} 
%%%%%%%%%%%%%%%%%%%%%%%%%%%%%%%%%%%%%%
%     EDUCATION
%%%%%%%%%%%%%%%%%%%%%%%%%%%%%%%%%%%%%%

\section{Opleidingen} 

\runsubsection{Hanze hogeschool}
\descript{| Groningen, }
\descript{HBO Bachelor Informatica}
\location{Spec. in Software Engineering}
2010 - 2014 
\sectionsep

\runsubsection{Drenthe College}
\descript{| Emmen }
\descript{MBO  specialist Applicatieontwikkelaar}
\location{2007 - 2010}
\sectionsep

%%%%%%%%%%%%%%%%%%%%%%%%%%%%%%%%%%%%%%
%     Internships
%%%%%%%%%%%%%%%%%%%%%%%%%%%%%%%%%%%%%%

\section{Stages}
\runsubsection{Topicus Finan B.V.}
\descript{| Software Engineer }
\location{September 2013 – Februari 2014 | Zwolle}
Een opdracht voor het maken van een IOS applicatie die monitoring toont van de huidige backoffice applicatie heeft er voor gezorgt dat ik bij Topicus begon. 
De REST backend was geschreven in Java met behulp van het PlayFramework. 
De frontend was geschreven in ObjectiveC met als doel device de Ipad.
\sectionsep

\runsubsection{Inversive Media}
\descript{| Software Developer }
\location{February 2013 – July 2013 | Groningen}
Bij Inversive Media heb ik een stage opdracht gedaan met de data van ‘Basisregistraties Adressen en Gebouwen’. 
De opdracht was het importeren en vervolgens kunnen ontsluiten van de data via REST.
Het ging hier om een dataset van 16GB aan XML bestanden. 
Deze werden geïmporteerd aan de hand van een script in ElasticSearch. 
Vervolgens kon de backend deze set gebruiken om verschillende mogelijkheden van zoeken te ondersteunen.
\sectionsep

\end{minipage}
\hfill
\begin{minipage}[t]{0.40\textwidth} 
%%%%%%%%%%%%%%%%%%%%%%%%%%%%%%%%%%%%%%
%     SKILLS
%%%%%%%%%%%%%%%%%%%%%%%%%%%%%%%%%%%%%%

\section{Skills}
\subsection{Programeertalen}

\begin{tabular}{lll}
\custombold{Java}			& 4 Jaar 	& \progress{0.8} \\
\custombold{TypeScript}   	& 2 Jaar	& \progress{0.75} \\
\custombold{Bash/Zsh}   		& 4 Jaar	& \progress{0.7} \\
\custombold{\LaTeX}   		& 2 Jaar	& \progress{0.6} \\
\custombold{Scss}  	 		& 2 Jaar	& \progress{0.6} \\
\custombold{Python}		& 1 Jaar	& \progress{0.6} \\
\end{tabular}\\
\sectionsep

\subsection{Frameworks \& Platforms}
\begin{tabular}{lll}
\custombold{Jenkins 2.0}   	& 3 Jaar	& \progress{0.9} \\
\custombold{GIT}  	 		& 5 Jaar	& \progress{0.85} \\
\custombold{Java EE}		& 4 Jaar	& \progress{0.8} \\
\custombold{Cucumber}   		& 3 Jaar	& \progress{0.8} \\
\custombold{Angular 2+}  	& 2 Jaar	& \progress{0.75} \\
\custombold{RxJs}			& 2 Jaar	& \progress{0.75} \\
\custombold{Html/Css}		& 6 Jaar	& \progress{0.75} \\
\custombold{Sass}			& 2 Jaar	& \progress{0.75} \\
\custombold{Docker CE}   		& 2 Jaar	& \progress{0.7} \\
\custombold{Hibernate}   		& 4 Jaar	& \progress{0.7} \\
\custombold{Fitnesse}		& ½ Jaar	& \progress{0.55} \\
\custombold{NgRx}			& ½ Jaar 	& \progress{0.55} \\
\end{tabular}\\
\sectionsep

\subsection{Operating systems}
\begin{tabular}{ll}
\custombold{Windows\ \ \ }   	& \progress{0.8} \\
\custombold{Linux}			& \progress{0.8} \\
\custombold{OS X}  		& \progress{0.7} \\
\end{tabular}\\
\sectionsep

\subsection{Overige}
GitHub \textbullet{} Jira \textbullet{} IntelliJ \\
Webstorm \textbullet{} Bitbucket \textbullet{} Confluence\\
Microsoft Office 
\sectionsep

\end{minipage} 


\end{document}  \documentclass[]{article}